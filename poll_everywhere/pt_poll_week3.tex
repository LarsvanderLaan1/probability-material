\documentclass[poll_tutorial_format]{subfiles}
\begin{document}
	\maketitle
		\setcounter{section}{2}
	\section{PT Week 3 ((Discrete) random variables)}
	
	\subsection{Set things up}
	\label{sec:set-things-up}
	
	
	
	\setcounter{theorem}{-1}
	
	\begin{exercise}
		Have you helped your neighbors to set up their polleverywhere app? 
		\begin{enumerate}
			\item Yes
			\item No
		\end{enumerate}
	\end{exercise}
	
	\subsection{Real questions}
	\label{sec:start-real-questions pt week 3}
	\begin{exercise}
		Suppose we have a sample space $S$ with a subset/event $A$. Denote $X$ the random variable such that $X(s)=1$ for $s\in A$ otherwise $0$. 
		Which one of these statements could be incorrect:
		\begin{enumerate}
			\item $\{X=1\}=A$
			\item $\{X=0\}=A^c$
			\item $\P{X\geq 0}=1$
			\item $\P{X=1}=\P{A}$
			\item $\P{X>=1}=1$ 
		\end{enumerate}
	\end{exercise}


	\begin{exercise}
	(Coin tossing problem) A fair coin is flipped two times, consider the sample space $S=\{HH, HT, TH, TT\}$ ($H$ stands for heads, and $T$ stands for tails). Denote r.v. $X$ as the number of heads within the two tosses.
	Which of these statements is incorrect: 
	\begin{enumerate}
		\item $\{X=2\}=\{HH\}$
		\item $ \{X=1\}=\{TH\}$
		\item $\{f(X) =4\} =\{HH\} $ with $f(x)=2x$
		\item $ \P{X\in \{0,1,2\}} =1$
	\end{enumerate}
\end{exercise}

	
	
	\begin{exercise}
		(Coin tossing problem) A fair coin is flipped two times, $A$ represents the event that both tosses landed heads and $B$ represents the event that the first toss landed tails. Denote r.v. $X$ the number of heads within two tosses.
		Choose one of these answers that is incorrect: 
		\begin{enumerate}
			\item $\{X=2\}=A$
			\item $ \{X=1\}=B$
			\item $\{X^2 =4\} =\{X=2\} $
			\item $ \{X=0\} \subseteq B$
		\end{enumerate}
	\end{exercise}
	
	
	
	\begin{exercise}
			Which of these statements is incorrect:
		\begin{enumerate}
			\item For a discrete random variable, its support may contain infinite numbers, but at most countably infinite.
			\item The probability mass function
			(PMF) of a discrete r.v., X, is a function that maps events of type $\{X=a\}$  to real numbers in $[0,1]$.
			\item The support of $X$ is $a_1, a_2, a_3, \dots$, it is possible that $\P{X=a_{3}}=0$.
		\end{enumerate}
	\end{exercise}
	
	
		\begin{exercise}
		Which of these statements is incorrect:
		\begin{enumerate}
			\item A random variable is a function that maps each outcome in the sample space to a number.  
			\item CDF functions are right-continuous functions increasing from 0 to 1.
			\item The CDF function of r.v. $X$ is a function that map events of type $\{X\leq a\}$ to real numbers in $[0,1]$.
			\item When we talk about the distribution of a random variable, we are solely referring to its PMF function.
		\end{enumerate}
	\end{exercise}
	
 		
		
		\begin{exercise}
			Denote $X$ a discrete random variable following the Discrete Uniform distribution with support $\{-2,-1,1,2\}$.
			Which of these statements is incorrect:
				\begin{enumerate}
						\item The PMF function $\P{X=a}=1/4$ for $a\in \{-2,-1,1,2\}$ otherwise 0 describes the how the probability is distributed among events generated by $X$. 
						\item The support of $X^2$ is $\{1,4\}$
						\item The support of $f(X)$ with $f(x)=x^2$ is $\{-1,-2,1,2\}$  
						\item The CDF function of $X$ describes how the probability is distributed among events generated by $X$.  
						\item The first and the forth choices.
					\end{enumerate}
			\end{exercise}
	
	
	\begin{exercise}
		Let $X_1,\cdots , X_n$ follow independent Bernoulli distribution with the same probability of succes $1/2$, 
		Which of these statements is incorrect: 
		\begin{enumerate}
			\item The PMF of $X_i$ is $\P{X_i=1]=1/2$ and $\P{X_i=0}=1/2$ otherwise 0.
			\item The CDF of $X_i$ is $F(x)=0$ for $x<0$; $F(x)=1/2$ for $0\leq x <1$; and $F(x)=1$ for $x\geq 1$.  
			\item $\sum_{i=1}^n X_n$ follows  Bin(n,1/2).
			\item The support of $\sum_{i=1}^n X_n$ is $\{1,2,3,4,...,n\}$.
		\end{enumerate}
	\end{exercise}
	
	
	
	\begin{exercise}
		Which of these statements is incorrect: 
		\begin{enumerate}
			\item CDF functions are right continuous.
			\item If $x_1\leq x_2$, for an arbitrary CDF function, $F(x_1)\leq F(x_2)$.
			\item For an arbitrary CDF function $F$, $F(x)$ takes value between 0 and 1.
			\item For an arbitrary CDF function $F$, $\lim\limits_{x\downarrow a}F(x)=F(a)$ (or equivalently,  $\lim\limits_{x\rightarrow a^+}F(x)=F(a)$).
			\item $\lim\limits_{x\uparrow +\infty} F(x)=+\infty$. 
		\end{enumerate}
	\end{exercise}
	
	
	\begin{exercise}
		Let $X_1, X_2$ follow independent Bernoulli distribution with the same probability of succes $1/2$, $X_3=1-X_2$ and $X_4=\max\{X_2, X_3\}$. 
		Which of these statements may be incorrect: 
		\begin{enumerate}
			\item $\P{X_4=1}=1$.
			\item $X_1$ is independent from $X_3$ and $X_4$.
			\item $X_4$ is independent from $X_2$.
			\item The support of $\max\{X_1,X_2\}$ is $\{0,1\}$.
		\end{enumerate}
	\end{exercise}
	
	
	\begin{exercise}
		(Coin tossing problem) A fair coin is flipped two times independently, $X_i$ denote the number of heads of $i$th toss. 
		Which of these statements is incorrect: 
		\begin{enumerate}
			\item The support of $X_i$ is $\{0,1\}$.
			\item $\P{X_1=a|X_2=b}=\P{X_1=a}$ for arbitrary real numbers $a$ and $b$.		
			\item $X_i$ is independent from $X_1+X_2$
			\item $\P{X_1=a, X_2=b}=\P{X_1=a}\P{X_2=b}$ for arbitrary real numbers $a$ and $b$.		
		\end{enumerate}
	\end{exercise}
	 
	
	
	
	
	
	
\end{document}
