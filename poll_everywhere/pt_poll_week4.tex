\documentclass[poll_tutorial_format]{subfiles}
\begin{document}
	\maketitle
	\setcounter{section}{3}
	\section{PT Week 4 Expectation (cases with (Discrete) random variables)}
	
	\subsection{Set things up}
	\label{sec:set-things-up}
	
	
	
	\setcounter{theorem}{-1}
	
	\begin{exercise}
		Have you helped your neighbors to set up their polleverywhere app? 
		\begin{enumerate}
			\item Yes
			\item No
		\end{enumerate}
	\end{exercise}
	
	\subsection{Real questions}
	\label{sec:start-real-questions pt week 4}
	\begin{exercise}
		Suppose we have a sample space $S$ with a subset/event $A$. Denote $X$ the random variable such that $X(s)=1$ for $s\in A$ otherwise $0$ 
		Which one of these statements could be incorrect:
		\begin{enumerate}
			\item $EX=P(X=1)$.
			\item $EX=P(A)$.
			\item The support of $X$ is $\{0,1\}$.
			\item $P(X=0|A)=1$. 
		\end{enumerate}
	\end{exercise}
	
	
	\begin{exercise}
		(Coin tossing problem) A fair coin is flipped two times, consider the sample space $S=\{HH, HT, TH, TT\}$ ($H$ stands for heads, and $T$ stands for tails). Denote r.v. $X$ the number of heads within two tosses.
		Which of these statements is incorrect: 
		\begin{enumerate}
			\item $\E{X}=\sum_{i=0}^2 i\P{X=i}$.
			\item $\E{X^2}=\left(\sum_{i=0}^2 i\P{X=i} \right)^2$ 
			\item $\E{X}$ is a real number.
			\item $\E{X^2}=\sum_{i=0}^2 i^2P(X=i)$
			\item $\E{X^2}=\sum_{i=0,1,4,9} iP(X^2=i)$
		\end{enumerate}
	\end{exercise}
	
	
		
	\begin{exercise}
		(Coin tossing problem) A fair coin is flipped two times, consider the sample space $S=\{HH, HT, TH, TT\}$ ($H$ stands for heads, and $T$ stands for tails). Denote r.v. $X$ the number of heads within the two tosses, and $Y$ the number of tails within the two tosses.
		Which of these statements is incorrect: 
		\begin{enumerate}
			\item $\E{aX+b}=a\E{X}+b$ for arbitrary numbers $a$ and $b$ 
			\item $\E{X^2 +Y^2}=\E{X^2} +\E{Y^2}$
			\item $\E{aX+bY}=a\E{X} +b\E{Y}$
			\item $\E{X}=2-\E{Y}$ 		
			\item $X$ and $Y$ are independent.
		\end{enumerate}
	\end{exercise}
	
	
	\begin{exercise}
	Which of these statements is incorrect: 
		\begin{enumerate}
			\item The expected value of $X$ is a weighted average of the possible values that $X$ can take on, weighted by their probabilities
			\item The expectation of $X$, if exists, is nothing but a number.
			\item  The expectation of a discrete random variable, $X$, is defined as $\E{X}=\sum_{x\in \textit{Support}(X)} x\P{X=x}$.
			\item The expectation of any discrete random variable always exist.  
		\end{enumerate}
	\end{exercise}
	
	
	
	\begin{exercise}
		Which of these statements is incorrect (alternatively select the last choice if all are correct):
		\begin{enumerate}
			\item $\E{X}$ depends only on the distribution of $X$: If $X$ and $Y$ are discrete r.v.s with the same distribution, then $\E{X} = \E{Y}$ (if either side exists).
			\item If the support of $X$ is $a_1, a_2, a_3, \dots$, then $\E{X}=\sum_{i=1}^\infty a_i \P{X=a_i}$.
			\item If $X$ and $Y$ have different distributions, then they have different expected values. 
			\item One only needs to know CDF or PMF of a discrete random variable $X$ to calculate the expected value of $X$.
			\item All of the above are correct. 
		\end{enumerate}
	\end{exercise}
	
	
	\begin{exercise}
		Which of these statements is incorrect: 
		\begin{enumerate}
			\item $\E{e^X}=e^{\E{X}}$ for an arbitrary r.v. $X$.  
			\item $\E{X+Y}=\E{X}+\E{Y}$
			\item $\E{X+Y}^2 =\E{X^2} +\E{Y^2} +2\E{XY}$  
			\item $\V{X}=\E{X-\E{X} }^2$.  
		\end{enumerate}
	\end{exercise}
	
	
		
	\begin{exercise}
		Choose one of these answers that is incorrect:
		\begin{enumerate}
			\item $\V{X}=\E{X-\E{X} }^2$  
			\item $\V{X}=\E{X^2} -\left(\E{X} \right)^2$
			\item $\E{X^2} \geq \left(\E{X} \right)^2$  
			\item The variance of a r.v. $X$, denoted by $\V{X}$, could be a negative value.
		\end{enumerate}
	\end{exercise}
	
	
	\begin{exercise}
		Let $A$, $B$ and $C$ be events, and let $I_A$, $I_B$, $I_C$ be indicator r.v.s. for events $A$, $B$  and $C$ respectively. 
		Which of these statements might be incorrect: 
		\begin{enumerate}
			\item $\E{I_AI_BI_C}=\P{A}\P{B}\P{C}$.
			\item $\E{I_AI_BI_C}=\P{A\cap B\cap C}$.  
			\item The fact that $I_A+I_B+I_C \geq I_{A\cup B \cup C}= \max\{I_A, I_B, I_C\}$ implies that $\P{A}+\P{B}+\P{C} \geq \P{A\cup B \cup C}$.
			\item $\E{I_A}=\P{A}$.
		\end{enumerate}
	\end{exercise}
	 

	\begin{exercise}
		Let $X$ be a discrete r.v. and $g$ a function from $\mathbb{R}$ to $\mathbb{R}$.
		Which of these statements is incorrect: 
		\begin{enumerate}
			\item $\E{e^X}= \sum_{i: \P{X=a_i}>0} e^{a_i} \P{X=a_i}$.
			\item We only need to know the PMF or the CDF of a discrete r.v. $X$ to calculate its expectation and the expectation of $g(X)$.
			\item $\E{g(X)}$ is determined by the distribution of $X$.
			\item $\E{X-\E{X}}^2 = (\E{X}-\E{X})^2=0$.
		\end{enumerate}
	\end{exercise}
	
	
	\begin{exercise}
		Which of these statements might be incorrect: 
		\begin{enumerate}
			\item We always have $\E{X^2} \geq \E{X}^2$.
			\item The variance of $X$ is the expected value of $g(X)$ with $g(x)=(x-\mu)^2, \mu=\E{X}$. 
			\item The variance of $X$ describes how likely $X$ to take values around its expectation. The smaller the variance, the closer $X$ is likely to be about its expectation.
			\item $\V{X}=\E{X^2}-\E{X}^2$. 
			\item $\V{aX+b}=a^2\V{X}$.
			\item $\V{X+Y}=\V{X}+\V{Y}$.
		\end{enumerate}
	\end{exercise}
	
	 
	
	
	
\end{document}
