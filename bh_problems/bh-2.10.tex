\input{header.tex}

\setcounter{theorem}{9}
\begin{exercise}[BH.2.10]
\begin{solution}~
	\begin{enumerate}
		\item Use LOTP with extra conditioning on $A_{2}$, and then the conditional independence of $A_{3}$ and $A_{1}$ given $A_{2}$ or $A_{2}^{C}$ that is given in the question.
		\begin{align*}
			P(A_{3}|A_{1}) &= P(A_{3}|A_{1},A_{2})P(A_{2}|A_{1}) + P(A_{3}|A_{1},A_{2}^{C})P(A_{2}^{C}|A_{1})\\
			&= P(A_{3}|A_{2})P(A_{2}|A_{1}) + P(A_{3}|A_{2}^{C})P(A_{2}^{C}|A_{1})\\
			&= 0.8^2 + 0.3\cdot 0.2\\
			& = 0.7.
		\end{align*}
		And also,
		\begin{align*}
			P(A_{3}|A_{1}^{C}) &= P(A_{3}|A_{1}^{C},A_{2})P(A_{2}|A_{1}^{C}) + P(A_{3}|A_{1}^{C},A_{2}^{C})P(A_{2}^{C}|A_{1}^{C})\\
			&= P(A_{3}|A_{2})P(A_{2}|A_{1}^{C}) + P(A_{3}|A_{2}^{C})P(A_{2}^{C}|A_{1}^{C})\\
			&= 0.8 \cdot 0.3 + 0.3\cdot 0.7\\
			& = 0.45.
		\end{align*}
		\item  Using the law of total probability (with $A_{1}$ as conditioning event) and the results from (a), we find that
		\begin{align*}
			P(A_{3}) &= P(A_{3}|A_{1})P(A_{1}) +P(A_{3}|A_{1}^{C})P(A_{1}^{C})\\
			& = 0.7\cdot 0.75 + 0.45\cdot 0.25\\
			& = 0.6375.
		\end{align*}
	\end{enumerate}
\end{solution}
\end{exercise}

\input{trailer.tex}
