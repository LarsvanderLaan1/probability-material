\input{header.tex}

\setcounter{theorem}{12}
\begin{exercise}[BH.2.13]
	\begin{solution}~
		\begin{enumerate}
			\item For Company B's test, the probability that a random person in the population is diagnosed correctly is $0.99$, since $99 \%$ of the people do not have the disease. For a random member of the population, let $C$ be the event that Company A's test yields the correct result, $T$ be the event of testing positive in Company A's test, and $D$ be the event of having the disease. Then
			$$
			\begin{aligned}
				P(C) &=P(C \mid D) P(D)+P\left(C \mid D^c\right) P\left(D^c\right) \\
				&=P(T \mid D) P(D)+P\left(T^c \mid D^c\right) P\left(D^c\right) \\
				&=(0.95)(0.01)+(0.95)(0.99) \\
				&=0.95,
			\end{aligned}
			$$
			which makes sense intuitively since the sensitivity and specificity of Company A's test are both $0.95$. So Company B is correct about having a higher overall success rate.
			\item  Despite the result of (a), Company A's test may still provide very useful information, whereas Company B's test is uninformative. If Fred tests positive on Company A's test, Example 2.3.9 shows that his probability of having the disease increases from $0.01$ to $0.16$ (so it is still fairly unlikely that he has the disease, but it is much more likely than it was before the test result; further testing may well be advisable). In contrast, Fred's probability of having the disease does not change after undergoing Company's B test, since the test result is a foregone conclusion.
			\item Let $s$ be the sensitivity and $p$ be the specificity of A's new test. With notation as in the solution to (a), we have
			$$
			P(C)=0.01 s+0.99 p .
			$$
			If $s=p$, then $P(C)=s$, so Company A needs $s>0.99$.
			If $s=1$, then $P(C)=0.01+0.99 p>0.99$ if $p>98 / 99 \approx 0.9899$.
			If $p=1$, then $P(C)=0.01 s+0.99$ is automatically greater than $0.99$ (unless $s=0$, in which case both companies have tests with sensitivity 0 and specificity 1).
		\end{enumerate}
	\end{solution}
\end{exercise}

\input{trailer.tex}
