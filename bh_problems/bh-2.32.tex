\input{header.tex}

\setcounter{theorem}{31}
\begin{exercise}[BH.2.32] 
\begin{solution}~
	\begin{enumerate}
		\item You can calculate these probabilities by conditioning on the outcome of one of the two dice involved.
		\begin{align*}
			P(A>B)& = P(A>B|A=4)P(A=4) + P(A>B|A=0)P(A=0)\\
			&=1\cdot \frac{2}{3} + 0\\
			& = \frac{2}{3},\\
			P(B>C)& = \frac{2}{3},\\
			P(C>D)& = P(C>D|C=6)P(C=6) + P(C>D|C=2)P(C=2)\\
			&=1\cdot \frac{1}{3} + \frac{1}{2}\cdot\frac{2}{3}\\
			& = \frac{2}{3},\\
			P(D>A)&= P(D>A|D=5)P(D=5) + P(D>A|D=1)P(D=1) \\
			&= 1\cdot \frac{1}{2} + \frac{1}{3}\frac{1}{2}\\
			& = \frac{2}{3}.
		\end{align*}
		\item If $A>B$, then you know $A=4$. This doesn't tell you anything about the event $B>C$, so $A>B$ is independent of $B>C$. If $B>C$, then you know that $C=2$, so  $P(C>D|B>C)=P(D<2)=\frac{1}{2}$. We see that $P(C>D|B>C)\neq P(C>D)$, so $C>D$ is not independent of $B>C$.
	\end{enumerate}
\end{solution}
\end{exercise}
\input{trailer.tex}
