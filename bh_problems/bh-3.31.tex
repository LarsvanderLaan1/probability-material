\input{header.tex}
\setcounter{theorem}{30}
\begin{exercise}[BH.3.31]
\begin{solution}~
    \begin{enumerate}
        \item Denote by $X$ the number of correct guesses. It is helpful to think for a moment about a particular ordering, for example:
        \begin{align*}
            M,M,T,M,T,T
        \end{align*}
        Refer to these as the milk and tea locations. To get $k$ of the milk teas correct, the lady needs to select $k$ out of 3 milk locations and $3-k$ out of 3 tea locations. In total, she can select 3 out of 6 locations for her guess. So, we find that
        \begin{align*}
            P(X=k) = \frac{{3 \choose k}{3 \choose 3-k}}{{6 \choose 3}}.
        \end{align*}
        Notice that the particular ordering MMTMTT above is not material to the argument.
        \item  Denote by $L$ the claim of the lady and by $M$ the event that the cup is milk first. We are looking for the posterior odds, so
        \begin{align*}
            odds(M|L) = \frac{P(M|L)}{P(M^{C}|L)}.
        \end{align*}
        We first calculate the numerator
        \begin{align*}
            P(M|L) = \frac{P(L|M)P(M)}{P(L|M)P(M) + P(L|M^{C})P(M^{C})} = \frac{p_{1}\frac{1}{2}}{p_1\frac{1}{2} + (1-p_{2})\frac{1}{2}}.
        \end{align*}
        where we have $ P(L|M^{C})=1-p_{2}$ because the cup is tea first, which the lady would correctly identify with probability $p_{2}$. The probability that she mistakes this for milk first is then $1-p_{2}$. Some algebra now shows that
                        \begin{align*}
        odds(M|L) = \frac{P(M|L)}{1-P(M|L)} =\frac{p_{1}}{1-p_{2}}.
        \end{align*}
        where we use the axioms of probability and the fact that the conditional probability is also a probability.
    \end{enumerate}
\end{solution}
\end{exercise}

\input{trailer.tex}
