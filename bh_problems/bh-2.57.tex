\input{header.tex}

\setcounter{theorem}{56}
\begin{exercise}[BH.2.57]
\begin{solution}
    \begin{enumerate}
        \item Suppose $C_{1}$ is small, but has a high percentage of green gummy bears. $M_{1}$ is large, and has a lower percentage of green gummy bears. $C_{2}$ is large and has a higher percentage green gummy bears than $M_{2}$, but lower than $M_{1}$. $M_{2}$ is small. Because $C_{1}$ and $M_{2}$ are small, the percentage of green gummy bears after merging will be close to the percentage in $C_{2}$ for the $C_{1}+C_{2}$ mixture, and close to the percentage in $M_{2}$ for the $M_{1}+M_{2}$ mixture. Since the percentage in $M_{2}$ is higher than in $C_{2}$, the $M_{1}+M_{2}$ mixture will also have a higher percentage of green gummi bears. In numbers, $C_{1}: 9G,1R$. $M_{1}: 500G,500R$, $C_{2}: 300G,700R$, $M_{2}:1G,9R$. The percentage of green gummy bears in $C$ is 309/1010, while in $M$ it is 501/1010.
        \item Intuitively this relates to the Simpson's paradox because also in this exercise it is important to take into account the difference in relative and overall frequencies of the different events occurring.

        Let $A$ be the event of drawing a green gummi bear. Let $B$ be the event that you sample from a jar labelled $M$ and let $C$ be the event that you sample from a jar labelled $1$. Then $P(A|B,C)<P(A|C^C,C)$ and $P(A|B,C^C)<P(A|C^C,C^C)$ but $P(A|B)>P(A|B^C)$.
    \end{enumerate}
\end{solution}
\end{exercise}
\input{trailer.tex}
