\input{header.tex}
\setcounter{theorem}{25}
\begin{exercise}[BH.1.26]
\begin{solution}~
	\begin{enumerate}
		\item In the birthday problem, birthdays enter the room with replacement. Here survey respondents enter with replacement.
		\item From the analysis of the birthday problem, we know that it is much easier to look at the probability of no match ($A$).
		$$P(A) = 1\cdot \left(1-\frac{1}{10^6}\right)\left(1-\frac{2}{10^6}\right)\ldots \left(1-\frac{999}{10^6}\right).$$
		The probability that at least one person will be chosen more than once is equal to $1-P(A) \approx 0.393$.
	\end{enumerate}
\end{solution}
\end{exercise}

\input{trailer.tex}
