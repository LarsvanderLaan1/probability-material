\input{header.tex}

\setcounter{theorem}{47}

\begin{exercise}[BH.2.48]
\begin{solution}
	\begin{enumerate}
		\item Suppose, we current have $n-i$ for $i=1,\ldots,6$, then we have a probability of $\frac{1}{6}$ of achieving $p_{n}$. So
		\begin{align*}
			p_{n} = \frac{1}{6}\sum_{i=1}^{6}p_{n-i}
		\end{align*}
		Assuming we start at 0, it is reasonable to set $p_{0}=1$ and $p_{k}=0$ for $k<0$. 
		\item
		\begin{align*}
			p_{1} &= \frac{1}{6},\\
			p_{2} &= \frac{1}{6}(p_{0}+p_{1}) =\frac{1}{6}\bigg(1+\frac{1}{6}\bigg),\\
			p_{3} &= \frac{1}{6}(p_{0}+p_{1}+p_{2}) = p_{2}(1+\frac{1}{6})=\frac{1}{6}\bigg(1+\frac{1}{6}\bigg)^2,\\
			p_{4} & = \frac{1}{6}(p_{0}+p_{1}+p_{2}+p_{3})=p_{3}\bigg(1+\frac{1}{6}\bigg) = \frac{1}{6}\bigg(1+\frac{1}{6}\bigg)^3,\\
			p_{5} & = \frac{1}{6}(p_{0}+p_{1}+p_{2}+p_{3} + p_{4})=p_{4}\bigg(1+\frac{1}{6}\bigg) = \frac{1}{6}\bigg(1+\frac{1}{6}\bigg)^4,\\
			p_{6} & = \frac{1}{6}(p_{0}+p_{1}+p_{2}+p_{3}+p_{4}+p_{5})=p_{5}\bigg(1+\frac{1}{6}\bigg) = \frac{1}{6}\bigg(1+\frac{1}{6}\bigg)^5,\\
			p_{7} & = \frac{1}{6}(p_{1}+p_{2}+p_{3}+p_{4}+p_{5}+p_{6})=p_{6}\bigg(1+\frac{1}{6}\bigg)-\frac{1}{6}p_{0}  = \frac{1}{6}\bigg(1+\frac{1}{6}\bigg)^6-\frac{1}{6}.\\	
		\end{align*}
		In decimals, $p_{7}=0.2536$.
	\end{enumerate}
\end{solution}
\end{exercise}
\input{trailer.tex}
