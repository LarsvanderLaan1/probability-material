\input{header.tex}

\setcounter{theorem}{33}

\begin{exercise} [BH.2.34] 
\begin{solution}~
	\begin{enumerate}
		\item Suppose you have an accident in year 1. Then, you are more likely to be a bad driver, and hence, also more likely to have an accident in year 2 as well. This means that $P(B|A)\neq P(B)$ and hence, $A$ and $B$ are not independent.
		\item[34b.] We find this probability using Bayes' rule,
		\begin{align*}
			P(G|A^{C})& =\frac{P(A^{C}|G)P(G)}{P(A^{C})}\\ &=\frac{P(A^{C}|G)P(G)}{P(A^{C}|G)P(G) + P(A^{C}|G^C)P(G^C)}\\
			&= \frac{(1-p_{1})g}{(1-p_{1})g + (1-p_{2})(1-g)}.
		\end{align*}
		\item[34c.] We are given that given $G$, the events $A$ and $B$ are independent. Then, using the LOTP,
 		\begin{align*}
			P(B|A^{C})& = P(B|A^{C},G)P(G|A^{C}) + P(B|A^{C},G^{C})P(G^{C}|A^{C})\\
			&=P(B|G)P(G|A^{C}) + P(B|G^{C})P(G^{C}|A^{C})\\
			&=p_{1}\frac{(1-p_{1})g}{(1-p_{1})g + (1-p_{2})(1-g)} + p_{2}\left(1-\frac{(1-p_{1})g}{(1-p_{1})g + (1-p_{2})(1-g)}\right)\\
			&=\frac{p_{1}(1-p_{1})g + p_{2}(1-p_{2})(1-g)}{(1-p_{1})g + (1-p_{2})(1-g)}.
		\end{align*}
	\end{enumerate}
\end{solution}
\end{exercise}

\input{trailer.tex}
