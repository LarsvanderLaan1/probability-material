\input{header.tex}
\setcounter{theorem}{11}

\begin{exercise}[BH.1.12]
\begin{solution}~
	\begin{enumerate}
		\item There are 52 cards, 13 of which are dealt to Player A. The number of possible hands (when ordering within the hand is not important) is $N_{A}={52 \choose 13}$.
	 	\item First deal to Player A. Following the answer to a., there are $N_{A}={52 \choose 13}$ options. Then deal to Player B. There are 39 cards left, of which this player will receive 13. This can be done in $N_{B} ={39 \choose 13}$ ways. Then deal to Player C. There are 26 cards left, of which this player will receive 13. This can be done in $N_{C}={26\choose 13}$ ways. We give the remaining 13 cards to Player D. In total, we have $$N = N_{A}N_{B}N_{C} = {52 \choose 13}{39 \choose 13}{26 \choose 13}$$ possibilities.
	 	\item If we would put the 13 cards that Player A received back into the deck before dealing to Player B, then this would be the answer.
	\end{enumerate}
\end{solution}
\end{exercise}

\input{trailer.tex}
