\input{header.tex}
\setcounter{theorem}{24}
\begin{exercise}[BH.3.25]
\begin{hint}
	For part (b), use (a) and the fact that $X$ and $Y$ are integer-valued.
\end{hint}
\begin{solution}~
	\begin{enumerate}
	    \item Assume that $X$ is the number of heads for Alice, then $V=n-X$ is the number of tails and $V=n-X\sim \text{Bin}(n,1/2)$. Similarly, assume that $Y$ is the number of heads for Bob, then $W=n+1-Y$ is the number of tails and $W=n+1-Y\sim \text{Bin}(n+1,1/2)$. Of course, $P(X<Y)=P(V<W)$ since $V$ has the same distribution as $X$ and $W$ has the same distribution as $Y$, $X$ and $Y$ are independent and $V$ and $W$ are independent.
        \item Now
        \begin{align*}
        	P(X<Y) = P(n-X<n+1-Y) = P(X>Y-1) = P(X\geq Y).
        \end{align*}
        Also, since $X<Y$ and $X\geq Y$ are disjoint and their union has probability equal to one, we have $1=P(X<Y) + P(X\geq Y) = 2P(X<Y)$. We conclude that $P(X<Y)=1/2$. 
	\end{enumerate}
\end{solution}
\end{exercise}

\input{trailer.tex}
