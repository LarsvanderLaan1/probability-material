\input{header.tex}
\setcounter{theorem}{43}

\begin{exercise}[BH.1.44]
\begin{solution}
	The difference $\Delta = B-A=A^{C}\cap B$. We know that $B = (A\cap B)\cup ( A^{C}\cap B) = (A\cap B)\cup (\Delta)$, where $A\cap B$ is disjoint from $\Delta$. Since these are disjoint $P(B) = P(A\cap B) + P(\Delta)$. Using now that $A\subseteq B$, we have $A\cap B = A$. Hence, $P(B) = P(A) + P(\Delta)$. We conclude that $P(\Delta) = P(B)-P(A)$.
\end{solution}
\end{exercise}

\input{trailer.tex}
