\input{header.tex}
\setcounter{theorem}{9}
\begin{exercise}[BH.3.10]
\begin{solution}~
	\begin{enumerate}
		\item For a PMF to be valid, we require that $\sum_{n}p_{X}(n)=1$. In this case, the support of the PMF are the positive integers, so we get $\sum_{n=1}^{\infty}p_{X}(n)=1$. It is asked whether it is possible that $p_{X}(n) = \frac{c}{n}$. In this case, $\sum_{n=1}^{\infty}\frac{c}{n}$  diverges. We conclude there does not exist a valid PMF as suggested.
		\item To show that such a PMF exists, we simply assume that the PMF at $n$ is equal to $\frac{c}{n^2}$ for some $c>0$. Then,
		\begin{align*}
			\sum_{n=1}^{\infty}\frac{c}{n^2} = c\frac{\pi^2}{6}.
		\end{align*}
		So if we choose $c=\frac{6}{\pi^2}$, we have a PMF that is valid.
	\end{enumerate}
\end{solution}
\end{exercise}

\input{trailer.tex}
