\subsubsection{Challenge: Recourse models}


This exercise will given an example of how probability theory can pop up in OR problems, in particular in linear programs. It introduces you to the concept of \textit{recourse models}, which you will learn about in the master course Optimization Under Uncertainty. Disclaimer: the story is quite lengthy, but the concepts introduced and questions asked are in fact not very hard. We just added the story to make things more intuitive.

We consider a pastry shop that only sells one product: chocolate muffins.
Every morning at 5:00 a.m., the shop owner bakes a stock of fresh muffins, which he sells during the rest of the day.
Making one muffin comes at a cost of $c = \$ 1$ per unit.
Any leftover muffins must be discarded at the end of the day, so every morning he starts with an empty stock of muffins.

The owner has one question for you: determine the amount $x$ of muffins that he should make in the morning to minimize his production cost. Note that the owner never wants to disappoint any customer, i.e., he requires that $x \geq d$, where $d$ is the daily demand for muffins.

The problem can be formulated as a linear program (LP):
\begin{align}
    \min_{x \geq 0} \{ cx \ : \ x \geq d \}.
\end{align}
For simplicity, we ignore the fact that $x$ should be integer-valued.

\begin{exercise}
Determine the optimal value $x^*$ for $x$ and the corresponding objective value in case $d$ is deterministic.
\end{exercise}

Of course, in practice $d$ is not deterministic. Instead, $d$ is a random variable with some distribution. However, note that the LP above is ill-defined if $d$ is a random variable. We cannot guarantee that $x \geq d$ if we do not know the value of $d$.

You explained the issue to the shop owner and he replies: ``Of course, you're right! You know, whenever I've run out of muffins and a customer asks for one, I make one on the spot. I never disappoint a customer, you know! It does cost me $50 \%$ more money to produce them on the spot, though, you know.''

Mathematically speaking, the shop owner just gave you all the (mathematical) ingredients to build a so-called \textit{recourse model}. We introduce a \textit{recourse variable} $y$ in our model, representing the amount of muffins produced on the spot. Production comes at a unit cost of $q = 1.5 c = \$ 1.5$. Assuming that we know the distribution of $d$, we can then minimize the \textit{expected total cost}:

\begin{align}
    \min_{x \geq 0} \big\{ cx + \E{v(d,x)} \big\},
\end{align}
where $v(d,x)$ is the optimal value of another LP, namely the \textit{recourse problem}:
\begin{align}
    v(d,x) := \min_{y \geq 0} \{ qy \ : \ x + y \geq d \},
\end{align}
for given values of $d$ and $x$. The recourse problem can easily be solved explicitly: we get $y=d-x$ if $d \geq x$ and $y=0$ if $d < x$. So we obtain
\begin{align}
    v(d,x) = q (d - x)^+,
\end{align}
where the operator $(\cdot)^+$ represents the \textit{positive value} operator, defined as
\begin{equation}
    (s)^+ = \begin{cases}
    s &\text{if } s \geq 0,\\
    0 &\text{if } s < 0.
    \end{cases}
\end{equation}

\begin{exercise}
To get some more insight into the model, suppose (for now) that $d \sim U\{10, 20\}$.
Solve the model, i.e., find the optimal amount $x^*$.
\begin{hint}
First, compute the value of $\E{v(d,x)}$ as a function of $x$. Then find the optimal value of $x$.
\end{hint}
\end{exercise}

\begin{exercise}
 What is the expected recourse cost (expected cost of on-the-spot production) at the optimal solution $x^*$, i.e., compute $\E{v(d,x^*)}$?
\end{exercise}

To solve the model correctly, we need the true distribution of $d$. We learn the following from the shop owner: ``My granddaughter, who's always running around in my shop, is a bit data-crazy, you know, so she's been collecting some data. I remember her saying that `the demand from male and female customers are both approximately normally distributed, with mean values both equal to $10$ and standard deviations of $5$'. She also mentioned something about correlation, but I don't remember exactly, you know. It was either almost $1$ or almost $-1$. I hope this helps!''

Mathematically, we've learned that $d = d_m + d_f$, with $(d_m, d_f) \sim \mathcal{N}(\mu, \Sigma)$, where $\mu = (\mu_m, \mu_f) = (10,10)$ and $\Sigma_{11} = \sigma_m^2 = \Sigma_{22}  = \sigma_f^2 = 5^2 = 25$. Finally, $\Sigma_{12} = \Sigma_{21} = \cov{d_m, d_f} = \rho \sigma_m \sigma_f = 25 \rho$. Also, we know that either $\rho \approx 1$ or $\rho \approx -1$.

\begin{exercise}
 Calculate $x^*$ and the corresponding objective value for the case $\rho = -1$. (Do not read  $\rho=1$, this case is not simple.)
\end{exercise}

\begin{exercise}
Consider the two extreme cases $\rho = 1$ and $\rho = -1$.
In which case will the shop owner have lower expected total costs?
Provide a short, intuitive explanation.
\begin{hint}
You don't have to compute $x^*$ for the case where $\rho = 1$; this is not easy!
\end{hint}
\end{exercise}
