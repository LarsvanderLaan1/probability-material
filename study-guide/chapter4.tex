% arara: pdflatex: { shell: yes }
% arara: pythontex: {verbose: yes, rerun: modified }
% arara: pdflatex: { shell: yes }

\input{header.tex}
\chapter{Chapter 4: Exercises and remarks}
When we talk about the expectation of a given random variable, we are talking about its average value (simply
a number): e.g., the expectation of a discrete real-valued r.v. (if it exists) is \textbf{a weighted average} of all possible values taken
by the random variable (some values are more likely to happen, so they contribute more to the average with larger
weights/probability values). The expectations of some specific functions of a given random variable are related to some
features of the given random variable: e.g., variance to describe its spread, fundamental bridge to get probability values via
the indicator functions. The fundamental bridge shows how closely the expectation and the probability are linked to each other, which are both measurements of uncertainty.
\\~\\
Expectation assigns random variables with their expected values (numbers if they exist). It is a linear and monotone
function from a set of random variables to a set of numbers. In our daily life, when we talk about the anticipated value of an investment with a (random) return in the future, we may use the expectation. Or you may wonder about the average raining hours in the coming day, again, the expectation. \\~\\ 
This is one of the most common concepts in your future studies. In decision theory/microeconomics, a rational agent making an optimal choice in the context of incomplete information is often assumed to maximize the expected value of their utility function; in prospect theory/behavior economics, people's choices are sometimes also modeled with expectations and researchers may try to elicit expectations in economics via laboratory experiments; while in econometrics, you would see the expectation idea hidden behind the concepts such as the bias of estimators.~
\\~\\
In the meantime, we continue with our journey into the discrete random variables. We have seen some special random
variables that process well-known distributions. This week, in particular, we see one important type of
discrete random variables: Poisson distributed random variables. We introduce Poisson distributed random variables as
a limit of a sum of many indicator random variables (with a success rate approaching zero, and number of indicators
approaching to infinity), and then we see some special properties of the Poisson distributed random variables. For example
the sum of two independent Poisson distributed random variables $A + B$ still follows a Poisson distribution and once given
the event that the sum is $n$, then the conditional probability of $A$ follows a Binomial distribution. In practice, the Poisson distribution is widely used
to approximate the distributions of occurrence of rare events (you will see its presence in actuarial science and extreme value
theory.).

\section{Expectation and variance}
\label{sec:section-4.1}

\begin{exercise}
	Prove the linearity of the expectation using the definition of expectation.
	\begin{solution}
		Let $X$ and $Y$ be random variables. Let $c$ be a constant. Then
		\begin{align*}
			\mathbb{E}(X+Y) & = \sum_{c\in \operatorname{supp}(X+Y)} c P(X+Y=c) \\
			&= \sum_{c\in \operatorname{supp}(X+Y)} c\left(\sum_{a \in \operatorname{supp}(X)} P(X=a, Y=c-a)\right) \\
			&= \sum_{a \in \operatorname{supp}(X)} \sum_{c\in \operatorname{supp}(X+Y)} c P(X=a, Y=c-a) \\
			&= \sum_{a \in \operatorname{supp}(X)} \sum_{c\in \operatorname{supp}(X+Y)}((a)+(c-a)) P(X=a, Y=c-a) \\
			&= \sum_{a \in \operatorname{supp}(X)} \sum_{b \in \operatorname{supp}(Y)}(a+b) P(X=a, Y=b) \\
			&= \sum_{a \in \operatorname{supp}(X)} \sum_{b \in \operatorname{supp}(Y)} a P(X=a, Y=b)+\sum_{a \in \operatorname{supp}(X)} \sum_{b \in \operatorname{supp}(Y)} b P(X=a, Y=b) \\
			&= \sum_{a \in \operatorname{supp}(X)} a \sum_{b \in \operatorname{supp}(Y)} P(X=a, Y=b)+\sum_{b \in \operatorname{supp}(Y)} b \sum_{a \in \operatorname{supp}(X)} P(X=a, Y=b) \\
			&= \sum_{a \in \operatorname{supp}(X)} a P(X=a)+\sum_{b \in \operatorname{supp}(Y)} b P(Y=b) \\
			&= E(X)+E(Y)
		\end{align*}
		and
		\begin{align*}
			E(cX) & = \sum_{x \in \text{supp}(cX)} x \P{cX = x} = \sum_{x\in \text{supp}(cX)} x \P{X = \frac{x}{c}} \\
			& = \sum_{y \in \text{supp}(X)} cy \P{X = y } = c\sum_{y \in \text{supp}(X)} y \P{X = y} \\
			& = c E(X).
		\end{align*}
	\end{solution}
\end{exercise}

\begin{exercise}
	Let $X$ be a random variable and let $Z=(X-E(X))^2$. Is $E(Z)$ smaller than, greater than or equal to $Var(X)$?
	\begin{solution}
		The two are equal: $E(Z) = E((X-E(X))^2) = Var(X)$.
	\end{solution}
\end{exercise}

\begin{exercise}\label{ex:chap04:03} 
	Prove that $E(XY) = E(X)E(Y)$ when $X,Y$ are independent (discrete) random variables.
	\begin{hint}
		Think about a new random variable which is a function of $X,Y$, such that $Z=g(X,Y)$ with $g(x,y)=xy$. We can then start with the definition of the expectation of $Z$, or we can use LOTUS $E(XY) = \sum_x \sum_y xy \P{X = x, Y = y}$ (these two ways lead to the same expected value). What does it mean for $X$ and $Y$ to be independent? 
	\end{hint}
	\begin{solution}
		Let $X$ and $Y$ be independent discrete random variables. Consider  $Z=g(X,Y)$ with $g(x,y)=xy$. Then $E(Z) = \sum_z z \P{Z = z}=\sum_z \sum_{x,y:xy=z} z \P{X = x, Y=y}$. As $X$ and $Y$ are independent, $\P{X = x, Y = y} = \P{X = x}\P{Y = y}$ for all values of $x$ and $y$. As such, $E(XY) = \sum_x \sum_y xy \P{X = x, Y = y} = \sum_x \sum_y xy \P{X = x} \P{Y = y} = \sum_x x \P{X = x} \sum_y y \P{Y = y} = E(X) E(Y)$.
	\end{solution}
\end{exercise}

\begin{exercise}\label{Exref{ex:chap04:03}}
	Calculate $E(XY) - E(X)E(Y)$ when $X,Z\sim \text{Bin}(n, p)$, $X,Z$ are independent and $Y = X + Z$.
	\begin{hint}
		Use the result of Ex \ref{ex:chap04:03}. 
	\end{hint}
	\begin{solution}
		Note that, because $X$ and $Z$ are independent, $E(XZ) = E(X) E(Z)$ (see Ex \ref{ex:chap04:03}). As such, $E(XY) = E(X(X + Z) = E(X^2 + XZ) = E(X^2) + E(X) E(Z)$. Moreover, $E(X)E(Y) = E(X) E(X + Z) = E(X)^2 + E(X) E(Z)$. As such, $E(XY) - E(X)E(Y) = E(X^2) + E(X) E(Z) - (E(X)^2 + E(X) E(Z)) = E(X^2) - E(X)^2 = V(X)$. Since $X\sim$Bin($n$,$p$), we know that $V(X) = np(1 - p)$. Hence $E(XY) - E(X)E(Y) = np(1 - p)$.
	
		Remark: if you are familiar with the concept of covariance, the problem becomes much easier. Note that $E(XY) - E(X)E(Y) = \text{Cov}(X,Y)$. Since covariance is linear and independent random variables have zero covariance, $E(XY) - E(X)E(Y) = \text{Cov}(X,Y) = \text{Cov}(X,X+Z) = \text{Cov}(X,X) + \text{Cov}(X,Z) = \text{Cov}(X,X) = V(X) = np(1 - p)$.
	\end{solution}
\end{exercise}

\begin{remark}
	Compare Ex \ref{ex:chap04:03} with Ex \ref{Exref{ex:chap04:03}}. We see that if $E(XY) - E(X)E(Y)$ is not zero, then $X$ and $Y$ are not independent. Indeed, non-zero values of $E(XY) - E(X)E(Y)$ imply dependence. However, the converse is not true: zero values of $E(XY) - E(X)E(Y)$ do \emph{not} imply independence! As an example, let $X$ be a random variable such that $\P{X = -1} = \P{X = 1} = \frac{1}{2}$ and let $Y = -X$. Then $X^2$ and $Y^2$ are not independent, yet $E(X^2Y^2) -E(X^2) E(Y^2) = 0$!    
\end{remark}

\begin{exercise}\label{ex:chap04:01}
	Let $X, Y \sim \text{Bin}(n,p)$ be independent.
	\begin{enumerate}
		\item Calculate $E(X)$.
		\item Calculate $E(2X)$.
		\item Calculate $E(X^2)$.
		\item Calculate $E(X+Y)$. 
		\item Calculate $E(XY)$.
		\item Is $E(X+Y)$ smaller than, equal to or greater than $E(2X)$? Why?
		\item Is $E(XY)$ smaller than, equal to or greater than $E(X^2)$? Why?
		\item Is $E(XY)$ smaller than, equal to or greater than $E(X)E(Y)$? Why?
	\end{enumerate}
	\begin{hint}
		To calculate $E(X)$ and $E(X^2)$, make use of the identity $k \binom{n}{k} = n \binom{n - 1}{k - 1}$ derived in Ex \ref{ex:chap01:11}.
	\end{hint}
	\begin{solution}
	Let $X, Y \sim \text{Bin}(n,p)$ be independent. To calculate $E(X)$ and $E(X^2)$, we make use of the identity $k \binom{n}{k} = n \binom{n - 1}{k - 1}$.
	\begin{enumerate}
		\item $E(X) = \sum_{k = 0}^n k \binom{n}{k} p^k (1 - p)^{n - k} = \sum_{k = 1}^n k \binom{n}{k} p^k (1 - p)^{n - k} $\\$= \sum_{k = 1}^n n \binom{n - 1}{k - 1} p^k (1 - p)^{n - k} = np \sum_{k = 1}^n \binom{n - 1}{k - 1} p^{k - 1} (1 - p)^{n - k} = np \sum_{i = 0}^m \binom{m}{i} p^j (1 - p)^{m - j} = np$
		\item By the linearity of the expectation operator, $E(2X) = 2E(X) = 2np$.
		\item $E(X^2) = \sum_{k = 0}^n k^2 \binom{n}{k} p^k (1 - p)^{n - k} = \sum_{k = 0}^n kn \binom{n - 1}{k - 1} p^k (1 - p)^{n - k}$\\$ = np \sum_{k = 1}^n k \binom{n - 1}{k - 1} p^{k - 1} (1 - p)^{(n - 1) - (k - 1)}$\\$ = np \left(\sum_{k = 1}^n (k - 1) \binom{n - 1}{k - 1} p^{k - 1} (1 - p)^{(n - 1) - (k - 1)} + \sum_{k = 1}^n \binom{n - 1}{k - 1} p^{k - 1} (1 - p)^{(n - 1) - (k - 1)}\right)$\\$ = np \left(\sum_{k = 0}^{n - 1} k \binom{n - 1}{k} p^{k} (1 - p)^{(n - 1) - k} + 1\right) = np((n - 1)p + 1) = n^2 p^2 + np(1 - p)$
		\item $E(X + Y) = E(X) + E(Y) = E(X) + E(X) = 2E(X) = 2np(1-p)$. Note that $E(X) = E(Y)$ as $X$ and $Y$ have the same distribution.
		\item By independence, $E(XY) = E(X)E(Y) = E(X)E(X) = E(X)^2 = n^2p^2(1 - p)^2$. (See Ex \ref{ex:chap04:03}.)
		\item $E(X + Y) = E(X) + E(Y) = E(X) + E(X) = 2E(X) = E(2X)$
		\item By independence, $E(XY) = E(X)E(Y) = E(X)E(X) = E(X)^2$. Notice that $V(X) = E(X^2) - E(X)^2 > 0$; as such $E(X^2) > E(X)^2 = E(XY)$.
		\item By independence, $E(XY) = E(X)E(Y)$. (See Ex \ref{ex:chap04:03}.)
	\end{enumerate}
	\end{solution}
\end{exercise}

\section{Negative hypergeometric and negative Binomial}
\label{sec:section-4.2}

\begin{exercise}\label{ex:chap04:07}
	A bag contains 13 sweet and 7 sour pieces of candy, which are randomly drawn one by one without replacement. What is the probability that you draw at least 2 of sweet pieces of candy before drawing the third piece of sour candy?
	\begin{hint}
		Let $X$ denote the number of pieces of sweet candy before drawing the third piece of sour candy. What is the distribution of $X$?
	\end{hint}
	\begin{solution}
		Let $X$ denote the number of pieces of sweet candy before drawing the third piece of sour candy. Then $X$ follows a \emph{negative hypergeometric} distribution with parameters $7$, $13$ and $3$; that is, $X \sim \text{NHGeom}(7, 13, 3)$. As such, the probability that $\P{X = k}$ is given by
		\begin{equation*}
			\P{X = k} = \frac{\binom{7}{2} \cdot \binom{13}{k}}{\binom{20}{k + 2}} \cdot \frac{5}{18 - k}
		\end{equation*}
		for $k = 0, 1, \hdots, 13$. It follows that
		\begin{align*}
			\P{X = 0} = \frac{\binom{7}{2} \cdot \binom{13}{0}}{\binom{20}{0 + 2}} \cdot \frac{5}{18 - 0} \approx 0.0307, \quad \P{X = 1} = \frac{\binom{7}{2} \cdot \binom{13}{1}}{\binom{20}{1 + 2}} \cdot \frac{5}{18 - 1} \approx 0.0704
		\end{align*}
		Hence $\P{X \geq 2} = 1 - \P{X = 0} - \P{X = 1} \approx 0.8989$.
	\end{solution}
\end{exercise}

\begin{exercise}
	Suppose again you draw pieces of candy one by one from a bag containing 13 sweet and 7 sour pieces of candy. This time, you draw \emph{with} replacement. What is the probability that you draw at least 2 pieces of sweet candy before drawing the third piece of sour candy? Is this probability higher or lower than the probability you found in Ex \ref{ex:chap04:07}?
	\begin{solution}
		For each draw, the probability of drawing a piece of sour candy is $\frac{7}{7 + 13} = 0.35$. Let $X$ denote the number of pieces of sweet candy before drawing the second piece of sour candy. Then $X$ follows a negative binomial distribution with parameters $3$ and $0.35$; that is, $X \sim \text{NBin}(3, 0.35)$. As such, the probability that $\P{X = k}$ is given by
		\begin{equation*}
			\P{X = k} = \binom{k + 2}{2} 0.35^3 0.65^k
		\end{equation*}
		for $k = 0, 1, \hdots$. It follows that
		\begin{align*}
			\P{X = 0} = \binom{0 + 2}{2} 0.35^3 0.65^0 \approx 0.0429 , \quad \P{X = 1} = \binom{1 + 2}{2} 0.35^3 0.65^1 \approx 0.0836
		\end{align*}
		Hence $\P{X \geq 2} = 1 - \P{X = 0} - \P{X = 1} \approx 0.8735$. This probability is higher than the probability found in Ex \ref{ex:chap04:07}.
	\end{solution}
\end{exercise}

\section{Indicator r.v.s and the fundamental bridge}
\label{sec:section-4.3}

\begin{exercise}
	Let $X$ and $Y$ be random variables. Let $Z$ be defined as
	\begin{equation}
		Z =
		\begin{cases}
			3 & \text{if } X < 2 \\
			X & \text{if } X \geq 2 \text{ and } Y < 0 \\
			Y & \text{otherwise}
		\end{cases}
	\end{equation}
	Express $Z$ in terms of $X$ and $Y$ using indicator functions.
	\begin{solution}
		$Z = 3 I_{X < 2} + (1 - I_{X < 2})(X I_{Y < 0} + Y (1 - I_{Y < 0}))$
	\end{solution}
\end{exercise}

\begin{exercise}
	Let $X$ be a non-negative integer-valued random variable. The \emph{survival function} of $X$ is given by $G_X(x) = 1 - F_X(x)$. Show that, if the expectation of $X$ exists, it equals $E(X) = \sum_{i = 1}^{\infty} G_X(i)$.
	\begin{hint}
		Write $X$ as the (infinite) sum of indicator random variables.
	\end{hint}
	\begin{solution}
		Write $X = I_{X \geq 1} + I_{X \geq 2} + \hdots = \sum_{i = 1}^{\infty} I_{X \geq i}$. By the fundamental bridge, $E(X) = \sum_{i = 1}^{\infty} \P{X \geq i} = \sum_{i = 1}^{\infty} (1 - \P{X < i}) = \sum_{i = 1}^{\infty} (1 - F_X(x)) = \sum_{i = 1}^{\infty} G_X(x)$.
	\end{solution}
\end{exercise}

\begin{exercise}
	There are $n$ visitors and $n$ balls with their names on respectively. Each visitor picks a ball randomly, then returns the ball. Denote $X$ the number of visitors picking the ball with their own name. Find $E(X)$.
	\begin{hint}
		Denote by $A_i$ the event that the $i$th visitor gets the correct ball. Use these events to $X$ in terms of indicator variables.
	\end{hint}
	\begin{solution}
		Denote by $A_i$ the event that the $i$th visitor gets the correct ball. Then $X = \sum_{i = 1}^n I_{A_i}$. By the fundamental bridge, $E(X) = \sum_{i = 1}^n \P{A_i}$. Note that $\P{A_i} = \frac{1}{n}$ by symmetry. Thus $E(X) = \sum_{i = 1}^n \frac{1}{n} = 1$.
	\end{solution}
\end{exercise}

\section{Expectation, variance, and CDF}
\label{sec:section-4.4}

\begin{exercise}
	Derive the CDF of $g(X)$ where $X\sim \text{Bern}(p)$ and 
	\begin{enumerate}
		\item $g(x)=2x$;
		\item $g(x)=x^2$;
		\item $g(x)=\ln (x+1)$ (just list the expression, no need to calculate the final result);
		%\item $g$ is the Bin($n$,$p$) CDF function;
		%\item $g$ is the Bern($s$) CDF function;
		%\item $g$ is the inverse of the Bern($s$) CDF function.
	\end{enumerate}
	\begin{solution}
		Let $X\sim \text{Bern}(p)$. The CDF of $X$ is given by
		\begin{equation*}
			F_X(x) =
			\begin{cases}
				0 & \text{for } x < 0 \\
				1 - p & \text{for } 0 \leq x < 1 \\
				1 & \text{for } x \geq 1
			\end{cases}.
		\end{equation*}
		\begin{enumerate}
			\item For $Y = g(X)$ where $g(x) = 2x$, $F_Y(y) = \P{Y \leq y} = \P{2X \leq y} = \P{X \leq y / 2} = F_X(y/2)$; therefore
			\begin{equation*}
				F_Y(y) =
				\begin{cases}
					0 & \text{for } y < 0 \\
					1 - p & \text{for } 0 \leq y < 2 \\
					1 & \text{for } y \geq 2
				\end{cases}.
			\end{equation*}
			\item For $Y = g(X)$ where $g(x) = x^2$, $F_Y(y) = \P{Y \leq y} = \P{X^2 \leq y} = \P{X \leq \sqrt{y}} = F_X(\sqrt{y})$; therefore
			\begin{equation*}
				F_Y(y) =
				\begin{cases}
					0 & \text{for } y < 0 \\
					1 - p & \text{for } 0 \leq y < 1 \\
					1 & \text{for } y \geq 1
				\end{cases}.
			\end{equation*}
			Note that $F_Y(y) = F_X(y)$ (in fact, $Y = X$).
			\item For $Y = g(X)$ where $g(x) = \ln(x + 1)$, $F_Y(y) = \P{Y \leq y} = \P{\ln(X + 1) \leq y} = \P{X + 1 \leq e^y} = \P{X \leq e^y - 1} = F_X(e^y - 1)$; therefore
			\begin{equation*}
				F_Y(y) =
				\begin{cases}
					0 & \text{for } y < 0 \\
					1 - p & \text{for } 0 \leq y < \ln 2 \\
					1 & \text{for } y \geq \ln 2
				\end{cases}.
			\end{equation*}
			%\item For $Y = g(X)$ where $g$ is the Bin($n$,$p$) CDF function, $F_Y(y) = \P{Y \leq y} = \P{g(X) \leq y} = \P{X \leq g^{-1}(y)} = F_X(g^{-1}(y))$; therefore
			%\begin{equation*}
			%	F_Y(y) =
			%	\begin{cases}
			%		0 & \text{for } 0 < y < (1 - p)^n \\
			%		1 - p & \text{for } (1 - p)^n \leq y < 1 - p^n \\
			%		1 & \text{for } 1 - p^n \leq y < 1
			%	\end{cases}
			%\end{equation*}
			%as $g^{-1}(y) = 0$ for $0 < y < (1 - p)^n$, $g^{-1}(y) \in (0, 1)$ for $(1 - p)^n \leq y < 1 - p^n$ and $g^{-1}(y) = 1$ for $1 - p^n \leq y < 1$.
			%\item For $Y = g(X)$ where $g$ is the Bern($s$) CDF function, $F_Y(y) = \P{Y \leq y} = \P{g(X) \leq y} = \P{X \leq g^{-1}(y)} = F_X(g^{-1}(y))$; therefore
			%\begin{equation*}
			%	F_Y(y) =
			%	\begin{cases}
			%		0 & \text{for } 0 < y < q \\
			%		1 & \text{for } s \leq y < 1
			%	\end{cases}
			%\end{equation*}
			%as $g^{-1}(y) = 0$ for $0 < y < s$ and $g^{-1}(y) = 1$ for $s \leq y < 1$.
			%\item For $Y = g(X)$ where $g$ is the inverse of the Bern($s$) CDF function, $F_Y(y) = \P{Y \leq y} = \P{g(X) \leq y} = \P{X \leq g^{-1}(y)} = F_X(g^{-1}(y))$; therefore
			%\begin{equation*}
			%	F_Y(y) =
			%	\begin{cases}
			%		0 & \text{for } y < 0 \\
			%		1 - p & \text{for } 0 \leq y < 1 \\
			%		1 & \text{for } y \geq 1
			%	\end{cases}
			%\end{equation*}
			%as $g^{-1}(y) = 0$ for $y < 0$ and $g^{-1}(y) = 1 - s$ for $0 \leq y < 1$ and $g^{-1}(y) = 1$ for $y \geq 1$.
	\end{enumerate}
	\end{solution}
\end{exercise}

\begin{exercise}
	Calculate $E(g(X))$, where $X\sim \text{Bern}(p)$ and 
	\begin{enumerate}
		\item $g(x)=2x$;
		\item $g(x)=x^2$;
		\item $g(x)=(x-E(X))^2$;
		\item $g(x)=\ln (x + 1)$;
		%\item $g$ is the Bin($n$,$p$) CDF function;
		%\item $g$ is the Bern($s$) CDF function;
		%\item $g$ is the inverse of the Bern($s$) CDF function.
	\end{enumerate}
	\begin{hint}
		Use the law of the unconscious statistician (LOTUS).
	\end{hint}
	\begin{solution}
	Let $X\sim \text{Bern}(p)$. Then:
		\begin{enumerate}
			\item Using linearity, for $g(x) = 2x$, $E(g(X)) = E(2X) = 2E(X) = 2(0 \cdot (1 - p) + 1 \cdot p) = 2p$;
			\item Using LOTUS, for $g(x) = x^2$, $E(g(X)) = E(X^2) = 0^2 \cdot (1 - p) + 1^2 \cdot p = p$;
			\item For $g(x) = (x - E(X))^2$, $E(g(X)) = E(x - E(X))^2 = V(X) = E(X^2) - E(X)^2 = p - p^2 = p(1 - p)$, using the results of the previous two parts;
			\item Using LOTUS, for $g(x) = \ln (x + 1)$, $E(g(X)) = E(\ln(X + 1)) = \ln 1 \cdot p + \ln 2 \cdot (1 - p) = (1 - p) \ln 2$.
			%\item Using LOTUS, for $g$ the $\text{Bin}(n,p)$ CDF function, $E(g(X)) = g(0) \cdot (1 - p) + g(1) \cdot p = \binom{n}{0} p^0 (1 - p)^n + p \cdot \binom{n}{1} p^1 (1 - p)^{n - 1} = (1 - p)^n + n p^2 (1 - p)^{n - 1}$;
			%\item Using LOTUS, for $g$ the $\text{Bern}(s)$ CDF function, $E(g(X)) = g(0) \cdot (1 - p) + g(1) \cdot p = s (1 - p) + p$;
			%\item Using LOTUS, for $g$ is the inverse of the Bern($s$) CDF function, $E(g(X)) = g(0) \cdot (1 - p) + g(1) \cdot p = 0 \cdot (1 - p) + 1 \cdot p = p$.
		\end{enumerate}
	\end{solution}
\end{exercise}

\begin{exercise}\label{ex:chap04:08}
Show that, for any random variable $X$, the variance of $X$ can be written as $V(X) = E(X^2) - (E(X))^2$. Start from the definition $V(X) = E(X - E(X))^2$.
	\begin{solution}
	Using the linearity of the expectation operator and the fact that $E(X)$ is a constant, we have $V(X) = E(X - E(X))^2 = E(X^2 - 2X E(X) + (E(X))^2) = E(X^2) - E(2X E(X)) + E((E(X))^2) = E(X^2) - 2 E(X) E(X) - (E(X))^2 = E(X^2) - 2 (E(X))^2 + (E(X))^2 = E(X^2) - (E(X))^2$.
	\end{solution}
\end{exercise}

\begin{exercise}
Show that, for any random variable $X$ and any constants $a$ and $b$, $V(aX + b) = a^2 V(X)$.
	\begin{solution}
	Using the result of Ex \ref{ex:chap04:03} and the linearity of the expectation operator, $V(aX + b) = E((aX + b) - E(aX + b))^2 = E((aX + b)^2) - (E(aX + b))^2 = E(a^2 X^2 + 2ab X + b^2) - (a E(X) + b)^2 = a^2 E(X^2) + 2ab E(X) + b^2 - (a^2 (E(X))^2 + 2ab E(X) + b^2) = a^2 (E(X^2) - (E(X))^2) = a^2 V(X)$.
	\end{solution}
\end{exercise}

\section{Poisson}
\label{sec:section-4.5}

\begin{exercise}
	Let $X \sim \text{Pois}(\lambda)$. Calculate the variance $V(X)$ of $X$.
	\begin{hint}
		Recall that $\sum_{k = 0}^{\infty} \frac{1}{k!} x^k = e^x$.
	\end{hint}
	\begin{solution}
		Let $X \sim \text{Pois}(\lambda)$. Recall that $\sum_{k = 0}^{\infty} \frac{1}{k!} x^k = e^x$. We have
		\begin{align*}
			E(X) & = \sum_{k = 0}^{\infty} k \P{Y = k} = \sum_{k = 0}^{\infty} k \cdot \frac{1}{k!} \lambda^k e^{-\lambda} = \sum_{k = 1}^{\infty} k \cdot \frac{1}{k!} \lambda^k e^{-\lambda} \\
			& = \lambda e^{-\lambda} \sum_{k = 1}^{\infty} \frac{1}{(k - 1)!} \lambda^{k - 1} = \lambda e^{-\lambda} \sum_{m = 0}^{\infty} \frac{1}{m!} \lambda^m = \lambda e^{-\lambda} \cdot \lambda e^{\lambda} \\
			& = \lambda
		\end{align*}
		and
		\begin{align*}
			E(X^2) & = \sum_{k = 0}^{\infty} k^2 \P{Y = k} = \sum_{k = 0}^{\infty} k^2 \cdot \frac{1}{k!} \lambda^k e^{-\lambda} = \lambda e^{-\lambda} \sum_{k = 0}^{\infty} k \cdot \frac{1}{(k - 1)!} \lambda^{k - 1} = \lambda e^{-\lambda} \sum_{k = 1}^{\infty} k \cdot \frac{1}{(k - 1)!} \lambda^{k - 1} \\
			& = \lambda e^{-\lambda} \left(\sum_{k = 1}^{\infty} (k - 1) \cdot \frac{1}{(k - 1)!} \lambda^{k - 1} + \sum_{k = 1}^{\infty} \frac{1}{(k - 1)!} \lambda^{k - 1}\right) \\
			& = \lambda e^{-\lambda} \left(\sum_{k = 2}^{\infty} (k - 1) \cdot \frac{1}{(k - 1)!} \lambda^{k - 1} + \sum_{k = 1}^{\infty} \frac{1}{(k - 1)!} \lambda^{k - 1}\right) \\
			& = \lambda e^{-\lambda} \left(\lambda \sum_{k = 2}^{\infty} \frac{1}{(k - 2)!} \lambda^{k - 2} + \sum_{k = 1}^{\infty} \frac{1}{(k - 1)!} \lambda^{k - 1}\right) \\
			& = \lambda e^{-\lambda} \left(\lambda \sum_{m = 0}^{\infty} \frac{1}{m!} \lambda^m + \sum_{m = 1}^{\infty} \frac{1}{m!} \lambda^m\right) \\
			& = \lambda e^{-\lambda} \left(\lambda e^{\lambda} + e^{\lambda}\right) \\
			& = \lambda^2 + \lambda \\
		\end{align*}
		It follows that $V(X) = E(X^2) - (E(X))^2 = \lambda^2 + \lambda - \lambda^2 = \lambda$.
	\end{solution}
\end{exercise}

\section{Connections between Poisson and Binomial}
\label{sec:section-4.6}

\begin{exercise}\label{ex:chap04:17}
	On a Monday morning, 400 people independently pass a street with 10 shops. Whenever a person passes a shop, they decide to visit the shop with probability 0.015. Assume each decision to enter a shop is independent of all others. Use the Poisson paradigm to approximate the probability that all shops are visited on Monday morning.
	\begin{solution}
		Let $X_{ij}$ be Bernoulli random variables that indicate whether person $j$ enters shop $i$, i.e. $X_{ij} = 1$ if person $j$ enters shop $i$ and $X_{ij} = 0$ otherwise. Let $Y_i$ denote the number of people that enter the $i$th shop, $i \in \{1, \hdots, 10\}$. According to the Poisson paradigm, the distribution of $Y_i$ can be approximated by a Poisson distribution with parameter $\lambda = E(Y_i) = E(\sum_{j = 1}^{400} X_{ij}) = 400 \cdot 0.015 = 6$. The $i$th shop is visited if and only if $Y_i \geq 1$. We have
		\begin{equation*}
			\P{Y_i \geq 1} = 1 - \P{Y_i = 0} = 1 - \frac{e^{-6} 6^0}{0!} = 1 - e^{-6}.
		\end{equation*}
		Therefore, the probability that all shops are visited is $\prod_{i = 1}^{10} \P{Y_i \geq 1} = (1 - e^{-6})^{10} \approx 0.9754$.
	\end{solution}
\end{exercise}

\begin{exercise}
	On a Monday morning, 400 people independently pass two shops. Whenever a person passes a shop, they decide to visit the shop with probability 0.015. Assume each decision to enter a shop is independent of all others. Given the fact that the two shops together have 11 visitors, approximate the probability that 7 people visit the first shop.
	\begin{solution}
		Let $X$ denote the number of people that enter the first shop and let $Y$ denote the number of people that enter the second shop. According to the Poisson paradigm, $X$ and $Y$ are approximately $\text{Pois}(6)$ distributed (see Ex \ref{ex:chap04:17}). Hence $X | X + Y = 11$ is approximately $\text{Binom}(11, 0.5)$ distributed. Hence $\P{X = 7| X + Y = 11} \approx \binom{11}{7} 0.5^{11} \approx 0.1611$.
	\end{solution}
\end{exercise}

\section{Law of the unconscious statistician (LOTUS)}
\label{sec:section-4.7}

 \begin{exercise}\label{eq:determin 1}
 	Use the fact that $E(X-EX)^2$ is always non-negative (not a surprise, the ``average'' of a non-negative $Y=(X-EX)^2$ r.v. is non-negative) to prove that 
 	$$\frac{1}{n}\sum_{i=1}^n x_i^2 \geq \left(\frac{1}{n}\sum_{i=1}^n x_i \right)^2,$$
 	where $x_i$'s are real numbers.
	\begin{hint}
		Consider a discrete uniform distributed $X$ with support $x_i, i=1,\cdots, n$.
	\end{hint}
	\begin{solution}
		For the numbers $\{x_1,\dots,x_n\}$, define a random variable $X$ that takes each value with equal probability. That is $\P{X=x_i}=\frac{1}{n}$ for all $i$. Then from the given inequality, the decomposition of the variance and LOTUS we have $0\leq E(X-E(X))^2=E(X^2)-(E(X))^2=\left(\sum_{i=1}\frac{1}{n}x_i^2\right)- \left(\sum_{i=1}^n\frac{1}{n}x_i\right)^2$. Rearranging gives the desired result.
	\end{solution}
\end{exercise}

 \begin{exercise}\label{eq:determin 2}
	Use the fact that $E(X-EX)^2=EX^2-(EX)^2$ to prove that 
	$$\frac{1}{n}\sum_{i=1}^n x_i^2 - \left(\frac{1}{n}\sum_{i=1}^n x_i \right)^2=\frac{1}{n}\sum_{i=1}^n\left(x_i- \frac{1}{n}\sum_{i=1}^n x_i \right)^2,$$
	where $x_i$'s are real numbers.
	\begin{hint}
		Consider a discrete uniform distributed $X$ with support $x_i, i=1,\cdots, n$.
	\end{hint}
	\begin{solution} 
		For the numbers $\{x_1,\dots,x_n\}$, define a random variable $X$ that takes each value with equal probability. That is $\P{X=x_i}=\frac{1}{n}$ for all $i$. Then from LOTUS we have $E(X-EX)^2=\sum_{i=1}^n\frac{1}{n}(x_i-\sum_{i=1}^n\frac{1}{n}x_i)^2$ and $E(X^2)-(E(X))^2=\left(\sum_{i=1}\frac{1}{n}x_i^2\right)- \left(\sum_{i=1}^n\frac{1}{n}x_i\right)^2$. The desired result then follows from the given equation.
	\end{solution}
\end{exercise}
\begin{remark}
	Ex \ref{eq:determin 1} and \ref{eq:determin 2} use properties of random variables (\textbf{stochastic} item) to prove \textbf{deterministic} inequalities/identities. This is not a surprise, as probabilities or random variables are defined with ``seemingly-deterministic'' concepts such as functions/mappings, and so they should also reflect some of these ``deterministic'' characters. The following Ex \ref{eq:determin 3}, asks you to use a deterministic inequality to prove a ``stochastic'' property.
\end{remark}

 \begin{exercise}\label{eq:determin 3}
	Use the fact that  
	$$\frac{1}{n}\sum_{i=1}^n x_i^2 \geq \left(\frac{1}{n}\sum_{i=1}^n x_i \right)^2,$$
	where $x_i$'s are real numbers and equality hods only when all $x_i$'s are equal, to prove (1) 
	$$P\left(\frac{1}{n}\sum_{i=1}^n X_i^2 \geq \left(\frac{1}{n}\sum_{i=1}^n X_i \right)^2\right)=1,$$
	where $X_i$'s are real random variables; and (2)
	$$P\left(\frac{1}{n}\sum_{i=1}^n Y_i^2 > \left(\frac{1}{n}\sum_{i=1}^n Y_i \right)^2\right)=1,$$
	where $Y_i$'s are \textbf{independent} real \textbf{continuous} random variables. 	
	\begin{hint}
		Suppose a deterministic inequality holds true for arbitrary values in a set $A$. Then the inequality should also holds for random variables with supports that are subsets of $A$, since whatever values they take, the inequality holds.  
	\end{hint}
	\begin{solution} 
		Suppose $\frac{1}{n}\sum_{i=1}^n  x_i^2  -  ( \frac{1}{n}\sum_{i=1}^n  x_i )^2 \geq 0$  holds true for any arbitrary constants $x_i$ 's, then if now we replace constant $x_i$ with r.v. $X_i$, we would have $P\left(\frac{1}{n}\sum_{i=1}^n  X_i^2  -  ( \frac{1}{n}\sum_{i=1}^n  X_i )^2 \geq 0 \right)=1$.
		Because for any arbitrary random events: $\cap \{X_i=s_i\}$ (suppose $X_i$ takes arbitrary real value $s_i$), we know $\frac{1}{n}\sum_{i=1}^n  s_i^2  -  ( \frac{1}{n}\sum_{i=1}^n  s_i )^2 \geq 0$. Notice we choose $s_i$ arbitrarily, so no matter what random events happen (or in other words, whatever elements from the underlying sample space and thus whatever numbers $X_i$'s map these elements into) we always have $\frac{1}{n}\sum_{i=1}^n  X_i^2  -  ( \frac{1}{n}\sum_{i=1}^n  X_i )^2 \geq 0$ conditional on these events.  Therefore, $\frac{1}{n}\sum_{i=1}^n  X_i^2  -  ( \frac{1}{n}\sum_{i=1}^n  X_i )^2 \geq 0$ holds true (no matter what random events occur) with probability one (you may also think about LOTP + Bayes'rule). \\~\\
		The equality hods true only when  all $x_i$'s are equal, note that when $Y_i$'s are \textbf{independent} real \textbf{continuous} random variables,
			$$P\left( Y_i=Y_j, \forall i,j \right)\leq P\left( Y_1=Y_2\right)=0,$$
			due to the fact that $Y_1-Y_2$ is also continuous random variable which equals any number with zero probability. 
		Therefore, the equality holds with probability one, and	
		$$P\left(\frac{1}{n}\sum_{i=1}^n Y_i^2 > \left(\frac{1}{n}\sum_{i=1}^n Y_i \right)^2\right)=1.$$
	\end{solution}
\end{exercise}
\begin{remark}
 Ex \ref{eq:determin 3}	links certainty (deterministic) and uncertainty (random) all together. Hopefully this enhances your understanding of r.v. which are nothing random but a special functions from the outcomes in the sample space to numbers.
\end{remark}

\begin{exercise}
	\label{1}$X,Y$ are two independent discrete random varaibles.	
	\begin{enumerate}
		\item \label{1a}Prove $\text{Var}(X+Y)=\text{Var}(X)+\text{Var}(Y)$.
		\item Provide a counter-example such that $Z_1,Z_2$ are two random varaibles and  $\text{Var}(Z_1+Z_2)>\text{Var}(Z_1)+\text{Var}(Z_2)$. Compare with the result you proved in the subquestion \ref{1}.(\ref{1a}), provide one intuitive explaination for the difference. 
	\end{enumerate}
\begin{solution}
	\begin{enumerate}
		\item Denote $X_0=X-E(X), Y_0=Y-E(Y)$, and  $\text{Var}(X+Y)=\text{Var}(X_0+Y_0)$ since we shift by a constant the varaince would remain the same. 
		\begin{align*}
			\text{Var}(X_0+Y_0)= E(X_0+Y_0)^2- \left(E(X_0+Y_0)\right)^2 = E(X_0+Y_0)^2.
		\end{align*}
		By LOTUS, we know 
		\begin{align}
			E(X_0+Y_0)^2& =_{} \sum_{x\in \text{Supp}(X_0),y\in \text{Supp}(Y_0)} (x+y)^2 P(X_0=x,Y_0=y)\nonumber\\
			&= \sum_{x\in \text{Supp}(X_0),y\in \text{Supp}(Y_0)} (x^2+y^2+2xy) P(X_0=x,Y_0=y)\nonumber \\
			&= \sum_{x\in \text{Supp}(X_0),y\in \text{Supp}(Y_0)} (x^2 ) P(X_0=x,Y_0=y)\nonumber\\&+\sum_{x\in \text{Supp}(X_0),y\in \text{Supp}(Y_0)} ( y^2 ) P(X_0=x,Y_0=y)\nonumber\\&+\sum_{x\in \text{Supp}(X_0),y\in \text{Supp}(Y_0)} ( 2xy) P(X_0=x,Y_0=y)   \label{11}
		\end{align}
		Note that by axioms of probability or LOTP and LOTUS:
		\begin{align*}
			& \sum_{x\in \text{Supp}(X_0),y\in \text{Supp}(Y_0)} (x^2 ) P(X_0=x,Y_0=y) \\
			=& \sum_{x\in \text{Supp}(X_0) } x^2 \sum_{ y\in \text{Supp}(Y_0)}   P(X_0=x,Y_0=y)\\
			= &  \sum_{x\in \text{Supp}(X_0) } x^2   P(X_0=x)\\
			=& E(X_0)^2
		\end{align*} 
		So we know 
		\begin{align}
			\sum_{x\in \text{Supp}(X_0),y\in \text{Supp}(Y_0)} (x^2 ) P(X_0=x,Y_0=y)=\text{Var}(X_0) \label{12}
		\end{align}
		since $\text{Var}(X_0)=E(X_0)^2-(E(X_0))^2$ for mean-zero $X_0$ ($E(X_0)=0$).
		Similarly,  we know 
		\begin{align}
			\sum_{x\in \text{Supp}(X_0),y\in \text{Supp}(Y_0)} (y^2 ) P(X_0=x,Y_0=y)=\text{Var}(Y_0) \label{13}
		\end{align}
		By independence,
		\begin{align}
			&\sum_{x\in \text{Supp}(X_0),y\in \text{Supp}(Y_0)} ( 2xy) P(X_0=x,Y_0=y) =  \sum_{x\in \text{Supp}(X_0)}\sum_{y\in \text{Supp}(Y_0)} ( 2xy) P(X_0=x)P(Y_0=y)\nonumber \\
			=& 2\sum_{x\in \text{Supp}(X_0)}  x   P(X_0=x) \sum_{y\in \text{Supp}(Y_0)}yP(Y_0=y)= 2 E(X_0)E(Y_0)=0 \label{14}
		\end{align}
		Equations (\ref{11})-(\ref{14}) togenter imply that $	\text{Var}(X_0+Y_0)=	\text{Var}(X_0)+	\text{Var}(Y_0).$
		\item 	Let $Z_1=Z_2=X$ and assume that $\text{Var}(X)>0$. Then 
		\begin{align*}
			\text{Var}(Z_1+Z_2)=	\text{Var}(2X) =4\text{Var}(X) > 2\text{Var}(X)=\text{Var}(Z_1)+\text{Var}(Z_2).
		\end{align*}
		The intuiation is when we sum up two independent r.v.'s, since they are independent, when one is large the other one could be small, and thus after the summation they may not be far away from the mean. If we add two highly dependent r.v.'s then they can be simutaneously large and small and thus increase the spread. 
	\end{enumerate}
	\end{solution}
\end{exercise}


\input{trailer.tex}

%%% Local Variables:
%%% mode: latex
%%% TeX-master: "study-guide.tex"
%%% End:
